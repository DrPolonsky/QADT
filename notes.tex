\documentclass[letterpaper,numbers=enddot]{scrartcl}

\usepackage{amsmath}
\usepackage{amssymb}

\title{Isomorphisms between algebraic datatypes}
\author{Andrew Polonsky, Ben Lenox}

\newcommand{\cat}{\mathbb{C}}
\newcommand{\op}{{op}}
\begin{document}

\maketitle

\section{Questions to be investigated}

\begin{enumerate}
  \item What is a strong isomorphism between ADTs?
  \item What does it mean for a functor to be a factor of another functor?
  \item A conception of a negative set in these terms.
  \begin{itemize}
    \item Universal $U$, and an ``integer set'' is a function from $U$ to $\mathbb{Z}$.
    \item A contravariant functor $- : \cat \to \cat^\op$ that splits the identity
    ($- \circ -^\op = id$)
    \item A solution to $E(X)=X$, obtained by taking $X = Y + 1$ for some type $Y$.
  \end{itemize}
  \item All polynomials functors with degree at least two will have linear factors.
  Can it solve arbitrary linear equation?
  Are all linear datatypes isomorphic to each other?  Are they generated by $S = -1$?
  \item Retractions between datatypes!
  \item Isomorphisms induced by initial algebra maps.
  \item Let $H(X) = 1 + X + X^3$, then $X$ contains sixth roots of unity as well as
  a square root of unity.  $M(X) = 1 + X + X^2$ contains fourth roots of unity;
  does this explain why we can ``divide $H(X)$ by $M(X)$''?

  Is this the reason why a solution to $M(X)=X$, when squared, yields a solution to $H(X)$?
  Why does this happen more generally?
\end{enumerate}

\section{To do:}
\begin{enumerate}
  \item Define an ADT.
  \item Define R(A?)DT: \emph{Rational (Algebraic?) Datatypes}
  \item Is every RADT an ADT?
\end{enumerate}

\section{Discussion}
Let $F$ be a functor, $X$ be the initial $F$-algebra.

By Lambek's Theorem, $F(X) \simeq X$.

For any polynomial $P(X)$ in $X$, we get $P(F(X)) \simeq P(X)$.

Relate this to the notion of strong isomorphism.

\end{document}
